\documentclass{article}
\usepackage[utf8]{inputenc}
\usepackage[a4paper, total={6.4in, 8.53in}]{geometry}
\usepackage{amsmath, tikz, amsfonts, bbm, mathrsfs, graphicx, amssymb, amsthm, hyperref, centernot, enumerate, bbm, xcolor, lmodern, mathdots, amsfonts, graphicx}

\title{MAT1856/APM466 Assignment 1}
\author{Yancheng Liu, Student \#: 1005038905}
\date{February, 2020}

\begin{document}

\maketitle

\section*{Fundamental Questions - 25 points}

\begin{enumerate}
    \item \hfill
    \begin{enumerate}
        \item Issuing bonds can help the government to control the money supply but also prevent inflation from simply printing more money. 

        \item For example, if the banks want to decrease the money supply, then they can increase the yield in order to let people deposit more and borrow less money, but at some time in the long run, banks will increase their money supply and decrease the yield rate back to normal, therefore, the long-term part of a yield curve might flatten. 

        \item Quantitative easing is letting the Fed purchase bonds from banks in order to increase the money supply and keep the interest rate at a low level, this can increase the economic activities within the US and help the US to grow the economy. The US Fed has used quantitative easing to recover their economy since the beginning of the COVID-19 pandemic.

    \end{enumerate}
    \item The coupon that I choose for this task is ‘CAN 9.25 Jun 1’, ‘CAN 0.25 Nov1’, ‘CAN 1.75 Mar 1’, ‘CAN 0.25 Aug 1’, ‘CAN 0.75 Feb 1’, ‘CAN 1.5 Sept 1’, ‘CAN 9.0 Jun1’, ‘CAN 0.5 Sept1’, ‘CAN 1.5 Jun1’, ‘CAN 0.5 Dec 1’. The first one has no coupon payment left until the maturity date, the second one has one more coupon payment until the maturity date, the third one has two more coupon payments until the maturity date and followed by the rules, the last one has nine more coupon payments. I choose these coupons depending on their maturity dates because it can help us better to analyze these coupons depending on different coupon payments.

    \item This is called Principal Component analysis. It is a method of dimensional reduction and helps us to reduce a large set of data into a small one which contains most of the information that we want. We need to compute the eigenvectors and eigenvalues of the covariance matrix to identify the principal components in the data. The first eigenvector tells us the correlation between the curves and the second eigenvector tells the steepness of the curve.

\end{enumerate}



\section*{Empirical Questions - 75 points} 

\begin{enumerate}
\setcounter{enumi}{3} 
    \item \hfill
    \begin{enumerate}
        \item  YTM curve figure
        \item Spot curve figure
        \item Forward curve figure
   
    \begin{figure}
\centering 
\subfigure[YTM curve]{
\label{Fig.sub.1}
\includegraphics[width=0.45]{YTM.png}}
\subfigure[name2]{
\label{Fig.sub.2}
\includegraphics[width=0.45\textwidth]{P+R_demand}}
\caption{Main name}
\label{Fig.main}
\end{figure}

\end{enumerate}
   
    \item 
    \item 
\end{enumerate}

\section*{References and GitHub Link to Code}
https://github.com/D0ngYang/APM466A1
\end{document}
